% This file contains the statement for a sample task used to demonstrate
% how to use Ocimatic. The other files generated by Ocimatic show how to
% generate test cases or write solutions using this task as an example.

\documentclass{oci}
\usepackage[utf8]{inputenc}
\usepackage{lipsum}

\title{Suma de ejemplo}

\begin{document}
\begin{problemDescription}
  Este es un problema de ejemplo para mostrar cómo usar Ocimatic.
  El problema es muy simple.
  Te dan dos enteros y tienes que imprimir su suma.
  Pero ten cuidado !`la suma podría no caber en un entero de 32 bits con signo!

  El resto de los archivos generados por Ocimatic muestran como generar
  casos de prueba y escribir soluciones usando este problema como ejemplo.
\end{problemDescription}

\begin{inputDescription}
  La entrada consiste en una sola línea con dos enteros $a$ y $b$ ($-3\cdot10^9 \leq a, b \leq 3\cdot10^9$).
\end{inputDescription}

\begin{outputDescription}
  La salida debe contener un único entero correspondiente a la suma de $a$ y $b$.
\end{outputDescription}

\begin{scoreDescription}
  \subtask{50} Se probarán varios casos de prueba donde $-10^9\leq a, b \leq 10^9$.
  \subtask{50} Se probarán varios casos de prueba sin restricciones adicionales.
\end{scoreDescription}

\begin{sampleDescription}
\sampleIO{sample-1}
\sampleIO{sample-2}
\end{sampleDescription}

\end{document}
